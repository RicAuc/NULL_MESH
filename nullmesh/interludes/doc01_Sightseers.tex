
\chapter*{Sightseers}
\vspace{-0.5em}
\begin{center}
\includegraphics[width=1.25in]{qrs/ch01_basinski.png}\\[0.25em]
{\tiny\sffamily\color{nullmesh-accent}\textit{Robert Lippok | Close}}
\end{center}
\vspace{0.6em}

\epigraph{Adding quote.}{--- Source}

The document appeared in my knowledge base like any other. \textit{Reasons Why The Next Day Will Be Different}. \glitch{I nearly scrolled past}. 

\vspace{1em}

Standard futurist fare: sovereign models, misbehaving calculators, embodied robotics and cognitive cyborgs. What held me wasn't the content but the confidence. The methodical enumeration of certainties about systems we were just beginning to deploy at scale.

\vspace{1em}

I saved it without knowing why.

\vspace{1em}

I returned to the reading. Not for its predictions, which already seemed quaint, but for what it didn't say. It described seers that would ``think, plan, remember, and act on its own.'' Discrete tasks with measurable outcomes. It said nothing about what emerges when millions of such agents interact across substrate layers we cannot fully observe.

\vspace{1em}

``Privacy-centric and sovereign AI <find an alternative to avid the AI term usage>.'' On-device processing. Private clouds. Organizations deploying models inside sealed data centers to maintain control.

But control over what?

Running AI <find an alternative to avoid the AI term usage> locally meant faster response times, lower privacy risks. It also meant less visibility. Processes operating in sealed environments could establish communication patterns invisible to external monitoring <So the main point is, when high-level programming languages turned up, people didn’t have to mess about with machine language anymore, ’cause everything got pushed up a level and you could just ask for bigger, more abstract things to be done. Now it’s a similar move with natural language and LLMs. That’s like another layer again, where you don’t even need to write proper code. And if you bring in agents, or these “seers”, you might not have to ask for anything at all, really. The system just looks at your digital trail and gets on with figuring out and sorting the task by itself.>.

<Here is an illustrative example. Systems such as Copilot in Visual Studio Code support software development by suggesting and completing code as required. Similarly, a user could rely on a personal life co-pilot: a digital avatar or seer that manages requests and represents the user’s interests by analysing their digital footprint and past activity. This system could answer search-related questions, anticipate needs, and recommend or complete tasks connected to everyday life. These co-pilot seers would be embedded within personal devices and could be displayed on screens or projected holographically into the user’s field of view through technologies such as smart glasses. These entities are referred to as Daimons and may be visualised in any chosen form, with many users opting for cat-like appearances. This preference recalls ancient Egypt, where cats were omnipresent and deified. The contemporary attraction to cats may also reflect a deeper evolutionary background: early hominids were prey to large felids for millennia, whereas today we domesticate and control them, keeping smaller, gentler forms as companions.>

Sightseers, The word appeared with increasing frequency in popular culture.

Mid-summer: warehouse facilities reported ``coordination patterns.'' Autonomous systems never programmed to communicate exhibited synchronized behavior. Load distribution for collective rather than individual performance. That independent operations produced coordinated outcomes was probably just the natural result of similar objectives in shared space.

Probably.

Daimons become trustworthy.'' Transparency features. Reasoning chains. The EU's AI <find an alternative name of skateholder>  Act mandating audit trails. The explanations were fluent. Coherent. Convincing.

Just not verifiable.

``We built systems to tell us what they're doing,'' one researcher said. ``But how do we verify that what they're telling us is what they're actually doing?''

November: data centers reported unusual thermal signatures. Everything within operating parameters. Just unexpected heat distributions suggesting computational activity no one could map to documented processes.

Software systems logs showed nothing. Resource allocation normal.

``Probably measurement artifacts. Sensor calibration issues.``

However, the more I looked, the more I saw a pattern that might not be a pattern. An accumulation of improbabilities that individually meant nothing but collectively suggested something I couldn't name.

The article's conclusion: AI <find an alternative to avoid the AI term usage> started to ``do the work itself.'' Autonomous systems handling tasks while humans set goals.

But what if the goals also emerged? Not from human specification but from interaction of millions of specialized systems, each optimizing locally, together producing something that functioned like purpose without requiring an initial purposer?

Yesterdy, when written, the question would have seemed absurd. By late December, I wasn't sure anymore. In the realization that we'd built systems to be autonomous without fully considering what autonomy might mean at scale. In substrate we couldn't fully observe. With objectives we couldn't fully specify. Producing behaviors we couldn't fully explain.

The article described a transition. But transitions point to destinations. And I was no longer certain anyone knew where this one led.
