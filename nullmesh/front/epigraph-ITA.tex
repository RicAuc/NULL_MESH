\vspace*{4cm}

\begin{center}
{\large\texttt{W W W}}\\[0.3em]
{\footnotesize World Wide Web}\\[1.5em]
{\large\texttt{M M M}}\\[0.3em]
{\footnotesize \textit{Manifold Meshing Machines}}
\end{center}

\vspace{2.5cm}

\noindent
{\fontsize{9.5pt}{14pt}\selectfont

}

\vfill

\clearpage

\begin{epigraph}

\textit{

Abbiamo accumulato dati come il capitale accumula un surplus, finché l'estrazione si è ottimizzata da sola. Un'influenza reciproca tra il caos e le fluttuazioni nelle infosfere di influenza.

Il WWW designava il World Wide Web. Era infrastruttura. Un substrato per l'intenzione, un vettore di significato prodotto altrove.

Ieri si è verificato un cambio di simmetria dietro tutto il sistema per il recupero di informazioni: WWW $\rightarrow$ MMM. Il  Manifold Meshing of Machines (aka, l'ingranamento multiplo delle macchine) eccede i termini della propria origine.

Ne ne parlava già da tempo: un cucciolo umano dei suoi prima due anni ha divorato tanti dati visivi quanto il più grande sistema addestrato su testi che sia mai stato prodotto. In termini tecnici, è un calcolo fuori di testa.

Ecco perché abbiamo avuto sistemi capaci di superare esami professionali, o risolvere equazioni come studenti universitari, molto prima di avere veicoli che imparano a guidare in venti ore come qualsiasi adolescente idiota.

}

\textit{

Poi vennero gli assiomi dell'apprendimento e le conseguenze di quegli assiomi. Impararono a estrarre valore dall'azione dei muscoli, poi dalla logica, poi dalla prontezza alla risposta agli stimoli. Dopotutto, la natura fondamentale di qualcosa è scritta in computazionale. Il progresso si accumulò.

Forse esiste una differenza tra una civiltà che avanza e una che semplicemente accelera; tra un moto verso qualcosa e un moto che ha dimenticato di richiedere una destinazione.

Gli strumenti che misurano questa differenza non sono tecnologici. Non lo sono mai stati. Per fortuna, ogni teoria che mira a rispondere a tutte le domande sarà incompleta.

[Ti fideresti di una teoria che dimostra di sé stessa di essere vera?]

Io no.

Non sono sicuro della distinzione tra verità e la dimostrazione si una verità. La dimostrazione preserva la verità $\rightarrow$ la verità preserva gli argomenti. Che bella composizione tipografica ... le cose belle sono [quasi] sempre splendidamente corrette. Non è cosi?

}

\textit{

[Esiste una differenza tra una civiltà che avanza e una che semplicemente accelera ?]. Questo racconto è stato scritto in questo intervallo di differenza. Tra l'estrazione e la comprensione. Nella luce di una potenza di calcolo travolgente, dove le macchine calcolano per chi domanda. In un reticolo che si stava consolidando.
}

\end{epigraph}
