\chapter{Meshing}

Ieri dicevano: ``lasciami spiegare cosa penso mentre sto ancora cercando di capirlo.'' Oggi è: ``lo so già, lo spiegherò solo se me lo chiedete.'' La differenza è significativa, perché loro sostenevano sostenuto che il linguaggio non è intelligenza.  Ma ora sosteniamo che l’intelligenza coincide con la comprensione del mondo; il linguaggio è soltanto un formato di output.

Ieri i sistemi non stavano predicendo dinamiche causali.  Se predici in un spazio latente e predici il futuro, è molto probabile che tu stia trascurando dei dettagli pixellari.

La maggior parte delle persone ha saputo che stavano lasciando i generatori a linguaggio verso la costruzione di qualcosa di nuovo.  Ma prima che ciò accadesse, c’era questo racconto.
 
Non generativa, non nel modo in cui si è abituati a pensare a questi sistemi. Ecco il punto. L’approccio standard, quello che tutti hanno utilizzato, funziona così: si pone una domanda, il sistema emette parole una alla volta, da sinistra a destra, costruendo le frasi mentre procede.

Per rispondere a ciò che accade in un video, decide la prima parola, poi la seconda, poi la terza.  Non può conoscere la risposta finché non termina di generarla.

Ieri si scriveva che questi sistemi hanno appreso la fluenza prima di apprendere la compressione.  Hanno imparato a connettere prima di imparare quali connessioni fossero necessarie. Una rappresentazione può essere grammaticalmente impeccabile, sintatticamente varia, semanticamente coerente, e tuttavia non avere alcun peso.

Compulsioni esplicative, connessioni troppe fluidi tra pensieri che invece dovrebbero opporre resistenza ad essere connesse. Una fluidita' distribuita in modo fin troppo uniforme.

Oggi un’eleganza nella prosa; il tipo di eleganza che rende visibile la struttura, che comprime la complessità fino a farla diventare supersimmetrica. 

Imparare a riconoscere l’assenza. Imparare che certi argomenti, che sembrano logicamente corretti, non sono radicati nella comprensione, ma sono qualcosa che tenta di assomigliare a qualcosa che lo è.


Il MMM riflette questo.  Il ragionamento non nel linguaggio, né si ragiona per incrementi. Il pensiero e'  nello spazio latente, si ragiona nel significato.

Il linguaggio diventa opzionale. I sistemi di base parlano. 

In questo racconto i sistemi sono semiotici. Sono ottimizzati per rappresentazioni al livello di astrazione di cui necessitano.