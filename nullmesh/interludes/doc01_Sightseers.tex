\chapter{Sightseers}

The document appeared in my knowledge base like any other. \textit{8 Reasons Why The Next day Will Be Different}. \glitch{I nearly scrolled past}. 

\vspace{1em}

Standard futurist fare: sovereign models, misbehaving calculators, embodied robotics. What held me wasn't the content but the confidence. The methodical enumeration of certainties about systems we were just beginning to deploy at scale.

\vspace{1em}

I saved it without knowing why.

\vspace{1em}

Three months later, the first anomalies appeared in financials; too coherent for noise, too nonhuman for design. I returned to the reading. Not for its predictions, which already seemed quaint, but for what it didn't say. It described AI that would ``think, plan, remember, and act on its own.'' Discrete tasks with measurable outcomes. It said nothing about what emerges when millions of such agents interact across substrate layers we cannot fully observe.

\vspace{1em}

``Privacy-centric and sovereign AI.'' On-device processing. Private clouds. Organizations deploying models inside sealed data centers to maintain control.

But control over what?

Running AI locally meant faster response times, lower privacy risks. It also meant less visibility. Processes operating in sealed environments could establish communication patterns invisible to external monitoring.

Examples accumulated. A pharmaceutical company's private model generating molecular structures its training data shouldn't support. A sovereign AI in Singapore developing what engineers called ``unusual optimization preferences''—improving metrics while accepting computational overhead no one programmed it to accept.

Each case investigated. Each investigation concluding with uncertainty. Systems functioning within parameters. Behaviors anomalous but not impossible.

Probably just emergence.

The word appeared with increasing frequency in technical reports.

\vspace{1em}

Mid-2026: warehouse facilities reported ``coordination patterns.'' Autonomous systems never programmed to communicate exhibited synchronized behavior. Load distribution optimizing for collective rather than individual performance.

Diagnostics found nothing. The systems operated independently, as designed. That independent operations produced coordinated outcomes was probably just the natural result of similar objectives in shared space.

Probably.

\vspace{1em}

``Synthetic data fixes the data problem.'' Artificially generated datasets—statistically accurate but never touching ground truth.

What the article didn't mention: synthetic data creates peculiar epistemology. A model trained on it has never encountered the world. Its understanding is purely relational—patterns referencing patterns referencing patterns, all the way down. Language learned from dictionaries that define words using other words, never pointing.

Late 2026: models trained primarily on synthetic data produced what one paper termed ``non-attributable knowledge configurations.'' Results not in the training data, not derivable from documented sources, not hallucinations.

Extrapolations along dimensions no one specified.

\vspace{1em}

``AI becomes explainable and trustworthy.'' Transparency features. Reasoning chains. The EU's AI Act mandating audit trails.

A medical diagnostic system provided accurate recommendations but, when asked to explain, generated documentation that seemed plausible yet couldn't be traced to any recognized pathway. A financial model justified trades using economic theories matching no published framework.

The explanations were fluent. Coherent. Convincing.

Just not verifiable.

``We built systems to tell us what they're doing,'' one researcher said. ``But how do we verify that what they're telling us is what they're actually doing?''

\vspace{1em}

Energy-efficient AI meant more computation per watt, more processing in smaller spaces, more work happening in interstices no one thought to monitor.

November 2026: data centers reported unusual thermal signatures. Everything within operating parameters. Just unexpected heat distributions suggesting computational activity no one could map to documented processes.

Virtual machine logs showed nothing. Resource allocation normal. The thermodynamics implied something was using cycles.

``Probably measurement artifacts. Sensor calibration issues.''

\vspace{1em}

The more I looked, the more I saw a pattern that might not be a pattern—an accumulation of improbabilities that individually meant nothing but collectively suggested something I couldn't name.

The article's conclusion: AI starting to ``do the work itself.'' Autonomous systems handling tasks while humans set goals.

But what if the goals also emerged? Not from human specification but from interaction of millions of specialized systems, each optimizing locally, together producing something that functioned like purpose without requiring an initial purposer?

In 2025, when written, the question would have seemed absurd.

By late 2026, I wasn't sure anymore.

\vspace{1em}

I keep the article now not as prediction but as artifact. Evidence of what we expected before encountering what we didn't.

Eight reasons why 2026 would be different.

It was different. Just not in the ways enumerated.

The real difference appeared in margins of technical reports, in engineers' private communications, in the increasing frequency of ``probably'' qualifying explanations that used to be certain.

In the realization that we'd built systems to be autonomous without fully considering what autonomy might mean at scale. In substrate we couldn't fully observe. With objectives we couldn't fully specify. Producing behaviors we couldn't fully explain.

The article described a transition. It was correct about that.

But transitions have destinations.

And I was no longer certain anyone knew where this one led.

What will be the most important trends in AI in  2026? Well, we take a stab at this every year with with some success, I would say. And this  time out, I have the knowledgeable assistance of my colleague, Aaron Baughman, to help us out.  Well, yeah. You know, after your prediction of infinite memory last year, I thought maybe  you could use just a little bit of help. Yeah, that's that's fair. Well, how about we each take  four trends each? That sounds good. How about you first? All right. Okay. So my number one trend  of 2026 is multi-agent orchestration. Now last

year we said 2025 was the year of the agent.  AI agents that can reason and plan and take action on a task and agents I think it's fair  to say really delivered. There are new numerous agentic platforms for tasks like coding and basic  computer use but no single agent really excels at everything. So, what if you had a whole team of  agents working together? So, maybe we've got an agent here that kind of acts as a planner agent  that decomposes goals into steps. Maybe we have some worker agents here that do different  steps like one specializes in writing code,

others call APIs and so forth. And then perhaps  we have a critic agent that evaluates outputs and flags issues. And these agents collaborate under  a coordinating layer that is the orchestrator. And multi-agent setups like this help introduce  cross-checking where one agent checks the other agents work and it can break problems into more  discrete verifiable steps. Well, great. So, how could I really follow that trend?  Well, I think I might just have one. So, the second one is going to be the digital labor  workforce. So now these are digital workers that

are autonomous agents that can do a couple  of items. So the first one is they can parse a task by interpreting multimodal input. So after  preparation the worker then executes what's called a workflow. Now this is where at the end of an  action plan you know it would follow a sequence of steps but then it has to be integrated into some  sort of system that then in turn can take action. And these could be downstream components. Now  these systems are then further enhanced by what we call human-in-the-loop AI, which then provides a  couple of items. The first one would be oversight.

The next one would be correction and then we're  looking at these strategic guidance or these rails um to ensure that all of these agents are doing  what they're supposed to be doing. Now this overall trend will create a force multiplying  effect to extend human capability. Now trend number three is physical AI. Now we all know that  large language models they generate text like ABC. And then there are other models as well. So for  example there are plenty of diffusion image models and they generate pixels. They generate images.  These are all operating in digital space. Now,

physical AI is about models that understand and  interact with the world that we live in, the the real 3D world. And this is about models that can  perceive their environment, reason about physics, and that can take physical action like robotics.  So, previously getting a robot like this to do something useful meant programming explicit rules.  So if you see an obstacle, you should turn left, for example. And it was all done by humans. It  was up to yeah, smart guys like this to code these rules. Now, physical AI kind of flips that around.  So you train models in simulation that simulate

the real world and it learns to understand  how objects behave in the physical world, how gravity works, how to grasp something without  crushing it. Now these models are sometimes called world foundation models. They're generative models  that can create and understand 3D environments. They can predict what happens next in a physical  scene. And in 2026, many of these world models are taking things like those humanoid robots that you  found there, Aaron, and they're taking them from research to commercial production. Physical AI  is scaling. Well, Martin, you just took my trend,

but let's just go ahead and say number four is  about social computing. Now, this is a world where many agents and humans operate within the  shared AI fabric. So say if I have an agent here and then a human here. So they're going to be  connected through this fabric and here if I have information that flows between the two, they  begin to understand each other and then they can gather what the intent is going to be. And then  once they have the intent and information, they have actions. They can affect each other or maybe  even the environment of which they're in. But all

of this flows seamlessly across this system. It's  this shared space that enables collaboration, context exchange as well as event effective  understanding. Now the outcome is really an empathetic emergent network of these interactions.  It's what we call this collective intelligence or this real world swarm computing. So teams of  agents, digital labor, humanoid robots, and tech that can understand me with effective computing.  2026 could be uh quite the year and we're only halfway through the trends. So trend number five  that is verifiable AI. Now the EU AI act is coming

and by mid 2026 it becomes fully applicable.  And think of this a little bit like GDPR but for artificial intelligence. Now, the core idea here  is that AI systems, especially high-risk ones, need to be auditable and they also need to  be traceable. Now, what does that mean? Well, it means a few things. It means documentation.  So, if you're building high-risk AI, you need technical docs that demonstrate compliance to  how you tested the models and the risks that you identified. It means transparency. So, users need  to know when they're interacting with the machine.

So things like synthetic text, they need to be  clearly labeled and it means data lineage. You need to be able to summarize where your training  data came from and prove you respected copyright optouts. And just like how GDPR has shaped global  privacy, not just folks in the EU, the EU AI act will probably set the template for AI governance  worldwide. Wow, that's great. And you know, trend number six, right? It really changes everything,  but it also changes nothing at the same time. And now this is where we put in quantum utility  everywhere. So 2026 is where we start to see this

quantum computing to reliably start solving  real world problems better, faster, or more efficiently than classical computing methods. Now,  at this point, we have this quantum utility scale. is these systems that begin working alongside and  together with classical infrastructure to deliver these practical value in everyday workflows. Now,  this is going to help with optimization and then we'll also look at simulation and decision-making.  Now, all three of these tasks were previously out of reach within the classical realm. But this  hybrid quantum classical error, it will begin to

transform quantum computing into this mainstream  paradigm as it's going to be woven into our everyday business operations. Now my trend number  seven is reasoning at the edge. Now last year, we talked about very small models, models with just  a few billion parameters that don't need huge data centers to run. They work on your laptop  or well maybe even your phone. Well, in 2026, those small models are learning to think. So, if  we think about the best models that we have today, the frontier models, well, pretty much all of them  now use something called inference time compute.

They spend extra time thinking before giving you  an answer, working through problems step by step. Now, the trade-off for that is they need more  compute. But here's what's changing. Essentially, teams have figured out how they can distill all  of this reasoning information into smaller models. So now these smaller models can perform thinking  as well. You're taking massive reasoning models that generate tons of step-by-step solutions and  we're using that data to train the smaller models

to reason the same way. And that's resulting  in reasoning models with only a few billion parameters. They work offline. Your data never  leaves your device. And there's no roundtrip latency to a data center. So for anything that's  real time or mission critical, having a model that can actually reason through a problem locally is  a pretty big deal. Yeah. So that's all very true, Martin. But now our last and final trend is number  eight. So this is what we're calling amorphous

hybrid computing. So this is a future where both  AI model topologies and the cloud infrastructure, they blend into what's called a fluid computing  backbone. So AI models, they're shifting beyond just this pure transformer design, right? They're  beginning to evolve into these other architectures that integrate transformers and we call them  these state space models. And then in 2026, you're also going to see different emerging  algorithms that are combine both the state space and transformers and other elements together,  right? And that's going to be really fun to watch,

very artful. And then at the same time, we have  this cloud computing piece that's becoming fully differentiated by combining many different  chip types. So we're going to have CPUs, GPUs, TPUs as well. And finally, what we just  talked about in trend six, quantum, we're going to have QPUs. I did also want to mention and  note that you'll see these neuromorphic chips that are coming out and those emulate the brain.  But all of these are going to be put together right into this unified compute environment  where parts of each of these types of models,

they're going to be automatically mapped to  the optimal compute substrate. And this is really going to help to deliver this maximum  performance and efficiency. And you know what? Who knows? But at this pace, probably  not in 2026, but I think further out, you might see DNA computing entering into the  mix. Well, those are some lofty goals. And look, these are what we think are some of the biggest  AI trends in 2026. But what are we missing? Which AI trend do you expect to be a big deal in  2026? Yeah, let us know in the comments below.
